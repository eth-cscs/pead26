The software tools -- Stackinator, uenv CLI, SLURM plugin and the pipeline implementations -- are quite mature and robust.
Much of our focus will be on quality of life improvements and small feature requests.

We have seen significant performance improvements from using \squashfs, with the corresponding reduction in IOPs that benefits other users on the system.
The use case of using \squashfs for user-installed software and datasets is promising, and is supported by uenv uenv tooling, as shown in~\sect{sec:usecase-squashfs}.
However, the workflow for doing this is ad-hoc.
We will add support for ``environment definitions'' in YAML files, that provide a list of uenv and additional \squashfs images to mount, with additional environment variables to set.

Currently the tools can't be deployed as a turnkey solution at another site.
Take, for example, the container registry support: this uses hard-coded address of the JFrog artifactory at CSCS.
However, all interactions with the registry are generic and could be used with any OCI container registry, for example DockerHub or the Github Container Registry.
If there is interest from other sites, we could make it easier to configure uenv for different sites with little effort.
